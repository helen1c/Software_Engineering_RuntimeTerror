\chapter{Zaključak i budući rad}
		
		Rezultat ovog projekta je izrađena web aplikacija  „Planinarski dnevnik“ koja ima implementirane razne funkcionalnosti koje olakšavaju život planinarima, kao što su stvaranje novih staza, planinarskih domova i događaja, uvid u već odrađene staze i događaje, pristup profilnim stranicama drugih planinara, mogućnost dodavanja prijatelja i slanja poruka te pregled popisa članova u vlastitoj planinarskoj zajednici. Aplikacija je dovršena nakon 16 tjedana, a u njezinu izradu bili su uključeni svi članovi tima.\\
		Na početku je problem bio neiskustvo većine članova tima  s odabranim alatima i programskim jezicima u kojima se izrađivala aplikacija, ali on se prebrodio ostvarenjem dobre komunikacije između članova, međusobnim pomaganjem, marljivim učenjem i trudom. Komunikacija se najvećim dijelom odvijala preko Microsoft Teamsa  i WhatsAppa. Izrada aplikacije podijeljena je u dvije faze. Prva faza je veći naglasak imala na izradi dokumentacije koja ide uz aplikaciju, dok je u drugoj fazi naglasak bio na programskoj implementaciji same aplikacije.\\ 
		U prvoj fazi tim se najprije upoznao, odabrao se vođa, dogovoren je okvirni plan kada bi koji dio aplikacije trebao biti dovršen i članovima su dodijeljeni prvi zadaci. U toj početnoj fazi rada također je osmišljen i izgled aplikacije i način rada većine funkcionalnosti koje aplikacija treba imati, što je članovima uvelike olakšalo kasniji rad. Izrada dokumentacije obuhvaćala je kreiranje obrazaca uporabe, sekvencijskih dijagrama, modela baze podataka i dijagrama razreda. “Podignute” su poslužiteljska i klijentska strana aplikacije  i napravljene su neke jednostavnije početne funkcionalnosti poput prijave i registracije.\\
		U drugoj fazi  krenulo se u implementaciju preostalih funkcionalnosti kojih je bilo poprilično mnogo, zbog čega je druga faza bila mnogo zahtjevnija i bio je potreban intenzivan rad svih članova tima. Međutim, upravo zbog prethodno dobro razrađenog plana i vizualnog osmišljavanja problemskih zadataka, nije bilo dvojbi oko načina na koji neki dio aplikacije treba raditi i izgledati pa je tim uspio implementirati veliki dio zahtjevanih funkcionalnosti na vrijeme. U dogovoru s asistentom neke unaprijed dogovorene funkcionalnosti su izbačene, a to su uglavnom funkcionalnosti vezane uz ulogu \textit{Dežurnog planinara} što bi bio i prvi sljedeći korak ako dođe do nastavka radi na aplikaciji. Na nekim dijelovima dokumentacije su napravljene male preinake i dodani su ostali potrebni UML dijagrami kao što su dijagram stanja, dijagram komponenti i dijagram razmještaja. Na kraju je odrađeno ispitivanje programskog rješenja pomoću Unit i Selenium testova.\\
		 
		Rad na ovom projektu je bio odlična prilika za stjecanje novih znanja i vještina koje će sigurno biti korisne svim članovima tima u budućnosti. Sudjelovanje na projektu obogatilo nas je za iskustvo rada u timu koji nije uvijek savršen, ali upravo zbog toga nas uči koliko je važna dobra komunikacija i vremenska usklađenost između njegovih članova. Jako smo zadovoljni s krajnjim rezultatom projekta - funkcionalna i oku ugodna aplikacija unatoč početnom neiskustvu većine članova tima. 
		