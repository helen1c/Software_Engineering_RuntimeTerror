\chapter{Specifikacija programske potpore}
		
	\section{Funkcionalni zahtjevi}
			
			\textbf{\textit{dio 1. revizije}}\\	
			
			\noindent \textbf{Dionici:}
			
			\begin{packed_enum}
				
				\item Naručitelj (FER)
				\item Razvojni tim
				\item Administrator				
				\item Planinari (korisnici aplikacije)
				\item Zaposlenici planinarskih domova
				
			\end{packed_enum}
			
			\noindent \textbf{Aktori i njihovi funkcionalni zahtjevi:}
			
			
			\begin{packed_enum}
				\item  \underbar{Neregistrirani/neprijavljeni korisnik (inicijator) može:}
				
				\begin{packed_enum}
					
					\item pregledati popis planinarskih staza
					\item pretraživati planinarske staze
					\begin{packed_enum}
						
						\item  prema zemljopisnom položaju 
						\item  prema prosječnom trajanju pješačenja za određenu stazu
						\item  prema zahtjevnosti planinarskog staze
				
					\end{packed_enum}
					\item pregledati popis planinarskih domova
					\item pretraživati planinarske domove
					\begin{packed_enum}
						\item prema dostupnoj infrastrukturi (voda, prenoćište, struja, hrana, internet)
						\item prema zemljopisnom položaju
					\end{packed_enum}
					\item se registrirati u sustav kao planinar tako što popuni formu za registraciju
				\end{packed_enum}
			
				\item  \underbar{Planinar (inicijator) može:}
				
				\begin{packed_enum}
					\item prijaviti se u sustav kao planinar
					\item upravljati vlastitim korisničkim računom
						\begin{packed_enum}
							
							\item  pregledati osobne podatke 
							\item  uređivati osobne podatke
							\item  ukloniti korisnički račun
							
						\end{packed_enum}
					\item pretraživati korisnike prema imenu i prezimenu
					\item ? pregledati profil drugog planinara (je li dovoljno samo da ga doda na popis prijatelja)
					\item poslati zahtjev za prijateljstvom (zahtjev za dodavanjem drugog planinara na popis vlastite planinarske zajednice)
					\item prihvatiti zahtjev za prijateljstvom
					\item pregledati pristigle zahtjeve za prijateljstvom
					\item vidjeti obavijest ako je drugi planinar prihvatio njegov zahtjev
					\item pregledati popis planinara u vlastitoj planinarskoj zajednici
					\item pregledati popis željenih planinarskih staza (favoriti)
					\item dodati planinarski izlet na popis željenih
					\item stvoriti vlastitu planinarsku stazu prema unaprijed određenom predlošku
					\item ocjenjivati stvorene planinarske staze drugih planinara
					\item prijaviti netočne i neprecizne informacije vezane uz planinarske staze
					\item pregledati staze koje je stvorio
					\item stvoriti događaj vidljiv na naslovnici na koji može
						\begin{packed_enum}
							
							\item  pozvati korisnike aplikacije s popisa vlastite planinarske zajednice 
							\item  pregledati popis ljudi koji dolaze na kreirani događaj
							
						\end{packed_enum}
					\item na naslovnici vidjeti nove objave korisnika s popisa vlastite planinarske zajednice
						\begin{packed_enum}
					
							\item  kreirani događaji
							\item  ostvareni bedževi
							
						\end{packed_enum}
					\item dodati ranije odrađene planinarske staze u arhivu
					\item dodati ranije posjećene planinarske domove u arhivu
					\item zatražiti potvrdu od dežurnog planinara da je bio u domu
					\item obzirom na svoju aktivnost zaraditi određeni bedž koji se prikazuje na njegovom profilu
					\item kontaktirati admina u slučaju potrebe za stvaranjem novog planinarskog doma ili promjene infrastrukture već postojećeg
				\end{packed_enum}
			
				\item  \underbar{Dežurni planinar (inicijator) može:}
				
				\begin{packed_enum}
					
					\item zatražiti ulogu dežurnog planinara u određenom planinarskom domu pri čemu mora priložiti odgovarajući dokaz 
					\item upravljati zaprimljenim zahtjevima za potvrdom posjeta u planinarskim domovima za koje je odgovoran
					\begin{packed_enum}
						
						\item pregledati zahtjeve
						\item potvrditi ili odbiti zahtjev
						
					\end{packed_enum}
					\item? pregledati domove za koje je dežuran
				\end{packed_enum}
			
				\item  \underbar{Administrator (inicijator) može:}
				
				\begin{packed_enum}
					
					\item obrisati korisničke račune
					\item upravljati zaprimljenim zahtjevima za dežurnog planinara
					\begin{packed_enum}
						
						\item pregled zahtjeva
						\item prihvaćanje i odbijanje zahtjeva
					
					\end{packed_enum}
					\item upravljati planinarskim stazama
						\begin{packed_enum}
							\item pregled prijavljenih netočnih i nepreciznih informacija vezanih uz objavljene planinarske staze
							\item uređivanje staza
							\item brisanje staza
						\end{packed_enum}	
					\item upravljati planinarskim domovima
					\begin{packed_enum}
						\item pregled prijavljenih netočnih i nepreciznih informacija vezanih uz objavljene planinarske domove
						\item uređivanje postojećih planinarskih domova
						\item stvaranje novog planinarskog doma
						\item brisanje planinarskih domova
					\end{packed_enum}
					\item pregledati poruke koje su poslali planinari
					
				\end{packed_enum}
	
				\item  \underbar{Baza podataka (sudionik):}
				
					\begin{packed_enum}
						
						\item komunicira s cjelokupnim sustavom
						\item pohranjuje sve podatke nužne za uspješno funkcioniranje sustava
						
					\end{packed_enum}
			\end{packed_enum}
			
			\eject 
			
			
				
			\subsection{Obrasci uporabe}
				
				\textbf{\textit{dio 1. revizije}}
				
				\subsubsection{Opis obrazaca uporabe}
					\textit{Funkcionalne zahtjeve razraditi u obliku obrazaca uporabe. Svaki obrazac je potrebno razraditi prema donjem predlošku. Ukoliko u nekom koraku može doći do odstupanja, potrebno je to odstupanje opisati i po mogućnosti ponuditi rješenje kojim bi se tijek obrasca vratio na osnovni tijek.}\\
					

					\noindent \underbar{\textbf{UC$<$broj obrasca$>$ -$<$ime obrasca$>$}}
					\begin{packed_item}
	
						\item \textbf{Glavni sudionik: }$<$sudionik$>$
						\item  \textbf{Cilj:} $<$cilj$>$
						\item  \textbf{Sudionici:} $<$sudionici$>$
						\item  \textbf{Preduvjet:} $<$preduvjet$>$
						\item  \textbf{Opis osnovnog tijeka:}
						
						\item[] \begin{packed_enum}
	
							\item $<$opis korak jedan$>$
							\item $<$opis korak dva$>$
							\item $<$opis korak tri$>$
							\item $<$opis korak četiri$>$
							\item $<$opis korak pet$>$
						\end{packed_enum}
						
						\item  \textbf{Opis mogućih odstupanja:}
						
						\item[] \begin{packed_item}
	
							\item[2.a] $<$opis mogućeg scenarija odstupanja u koraku 2$>$
							\item[] \begin{packed_enum}
								
								\item $<$opis rješenja mogućeg scenarija korak 1$>$
								\item $<$opis rješenja mogućeg scenarija korak 2$>$
								
							\end{packed_enum}
							\item[2.b] $<$opis mogućeg scenarija odstupanja u koraku 2$>$
							\item[3.a] $<$opis mogućeg scenarija odstupanja  u koraku 3$>$
							
						\end{packed_item}
					\end{packed_item}
				
					
				\subsubsection{Dijagrami obrazaca uporabe}
					
					\textit{Prikazati odnos aktora i obrazaca uporabe odgovarajućim UML dijagramom. Nije nužno nacrtati sve na jednom dijagramu. Modelirati po razinama apstrakcije i skupovima srodnih funkcionalnosti.}
				\eject		
				
			\subsection{Sekvencijski dijagrami}
				
				\textbf{\textit{dio 1. revizije}}\\
				
				\textit{Nacrtati sekvencijske dijagrame koji modeliraju najvažnije dijelove sustava (max. 4 dijagrama). Ukoliko postoji nedoumica oko odabira, razjasniti s asistentom. Uz svaki dijagram napisati detaljni opis dijagrama.}
				\eject
	
		\section{Ostali zahtjevi}
		
			\textbf{\textit{dio 1. revizije}}\\
		 
			 \textit{Nefunkcionalni zahtjevi i zahtjevi domene primjene dopunjuju funkcionalne zahtjeve. Oni opisuju \textbf{kako se sustav treba ponašati} i koja \textbf{ograničenja} treba poštivati (performanse, korisničko iskustvo, pouzdanost, standardi kvalitete, sigurnost...). Primjeri takvih zahtjeva u Vašem projektu mogu biti: podržani jezici korisničkog sučelja, vrijeme odziva, najveći mogući podržani broj korisnika, podržane web/mobilne platforme, razina zaštite (protokoli komunikacije, kriptiranje...)... Svaki takav zahtjev potrebno je navesti u jednoj ili dvije rečenice.}
			 
			 
			 
	